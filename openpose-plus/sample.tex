\documentclass[twoside,11pt]{article}

% Any additional packages needed should be included after jmlr2e.
% Note that jmlr2e.sty includes epsfig, amssymb, natbib and graphicx,
% and defines many common macros, such as 'proof' and 'example'.
%
% It also sets the bibliographystyle to plainnat; for more information on
% natbib citation styles, see the natbib documentation, a copy of which
% is archived at http://www.jmlr.org/format/natbib.pdf

% Available options for package jmlr2e are:
%
%   - abbrvbib : use abbrvnat for the bibliography style
%   - nohyperref : do not load the hyperref package
%   - preprint : remove JMLR specific information from the template,
%         useful for example for posting to preprint servers.
%
% Example of using the package with custom options:
%
% \usepackage[abbrvbib, preprint]{jmlr2e}

\usepackage{jmlr2e}

% Definitions of handy macros can go here

\newcommand{\dataset}{{\cal D}}
\newcommand{\fracpartial}[2]{\frac{\partial #1}{\partial  #2}}

% Heading arguments are {volume}{year}{pages}{date submitted}{date published}{paper id}{author-full-names}

\jmlrheading{1}{2000}{1-48}{4/00}{10/00}{meila00a}{Marina Meil\u{a} and Michael I. Jordan}

% Short headings should be running head and authors last names

\ShortHeadings{Learning with Mixtures of Trees}{Meil\u{a} and Jordan}
\firstpageno{1}

\begin{document}

\title{OpenPose-Plus: Flexible Real-time Human Pose Estimation}

\author{\name Marina Meil\u{a} \email mmp@stat.washington.edu \\
       \addr Department of Statistics\\
       University of Washington\\
       Seattle, WA 98195-4322, USA
       \AND
       \name Michael I.\ Jordan \email jordan@cs.berkeley.edu \\
       \addr Division of Computer Science and Department of Statistics\\
       University of California\\
       Berkeley, CA 94720-1776, USA}

\editor{Kevin Murphy and Bernhard Sch{\"o}lkopf}

\maketitle

\begin{abstract}%   <- trailing '%' for backward compatibility of .sty file
%This paper describes the mixtures-of-trees model, a probabilistic 
%model for discrete multidimensional domains.  Mixtures-of-trees 
%generalize the probabilistic trees of \citet{chow:68}
%in a different and complementary direction to that of Bayesian networks.
%We present efficient algorithms for learning mixtures-of-trees 
%models in maximum likelihood and Bayesian frameworks. 
%We also discuss additional efficiencies that can be
%obtained when data are ``sparse,'' and we present data 
%structures and algorithms that exploit such sparseness.
%Experimental results demonstrate the performance of the 
%model for both density estimation and classification. 
%We also discuss the sense in which tree-based classifiers
%perform an implicit form of feature selection, and demonstrate
%a resulting insensitivity to irrelevant attributes.
\end{abstract}

\begin{keywords}
  Pose Estimation, Computer Vision, Real-time Systems
\end{keywords}

\section{Introduction}

Human pose estimation is a core task in computer vision.
Its goal is to localise human anatomical key-points (e.g., elbow, wrist, etc.)
and use the detected key-points to infer the meaning of human pose. 
Estimating human pose has many useful applications, such as interactive gaming~\citep{x}, automated supermarket~\citep{x},
surveillance camera~\citep{x} and self-driving cars~\citep{x}.
These applications are often developed and tested
in high-performance servers, and deployed at energy-constrained 
edge and embedded devices. 

Many practitioners have shown the strong interests in applying 
pose estimation techniques in the real world.
They, however, report two major challenges: (1) \emph{Flexibility}.
It is challenging to customize state-of-the-art pose estimation
algorithms for applications. 
According
to applications and environments,
these algorithms should be adapted to use different
data pre-processing/post-processing modules~\citep{x1, x2}
and deep neural networks~\citep{x1, x2, x3}.
For example, a mobile developer 
wants to trade-off accuracy for
performance by replacing an high-accuracy
model (i.e., Inception~\citep{x}) with an memory-efficient model (i.e., MobileNet~\citep{x}), and further train the new model
with custom datasets;
(2) \emph{Real-time performance}. It is also challenging to achieve real-time data processing
required by most pose estimation applications.
These applications produce high-resolution images
at high rates (e.g., [....]). Processing such data at line rates 
is non-trivial for the
hardware (e.g., Jetson TX2) available on commodity edge devices~\citep{evidence}.

Though several pose estimation systems have recently become available,
none of them can provide the required flexibility 
and real-time performance at the same time. 
Oftentimes users have to choose either flexibility
or performance, and must compromise the other.
On one extreame, systems like OpenPose~\citep{x}
are designed for specific pose estimation tasks and hardware architectures.
They have monolithic optimised system components,
and do not have principle APIs to alter the choices of 
data modules and neural networks, making
them rigid and brittle to be adapted.
On the other extreme, users can quickly prototype
pose estimation applications~\citep{x1, x2}
using general-purpose 
CPU/GPU engines such as TensorFlow and PyTorch. 
The performance of these systems is ultimately
limited by their comprehensive usages of high-level dynamic
languages (i.e., Python) and the lack of designs for
efficient execution on embedded platforms.

To close this gap, we argue that: to achieve both
flexibility and real-time performance in 
human pose estimation, it is important
to co-design algorithm and their execution runtime.
To demonstrate this, we design and develop: OpenPose-Plus,
a flexible real-time pose estimation framework.
The design of OpenPose-Plus makes two key contributions: it proposes
(i) high-level Python APIs which allows users
to deeply customise the data modules
and neural networks used in training an pose estimation
algorithm, and (ii) a real-time runtime which
can automatically offload the trained algorithm
onto embedded hardware: TensorRT, and 
fully exploit the potential of such hardware by 
minimising data movement cost and achieving the highest possible
level of parallel usage of available computation devices (i.e., CPUs and GPUs). 

We evaluate the effectiveness of OpenPose-Plus 
using extensive application studies and
test-bed performance evaluation. 
Evaluation result show that the high-level APIs of OpenPose-Plus
are sufficient to implement challenging pose estimation
applications which are difficult to release today.
OpenPose-Plus is also able to process the data in real-time
on state-of-the-art embedded platforms. Specifically, it can
process [...] images per second on [...].
OpenPose-Plus was open-sourced in October 2018 under a 
Apache 2 licence. Since after, it has attracted numerous 
industry and academic users, and has a quickly growing use base on Github (700 stars by the date of submission). 

\section{Architecture and Runtime Design}

\section{High-level APIs and Applications}

\section{Conclusion}

% Acknowledgements should go at the end, before appendices and references

%\acks{We would like to acknowledge support for this project
%from the National Science Foundation (NSF grant IIS-9988642)
%and the Multidisciplinary Research Program of the Department
%of Defense (MURI N00014-00-1-0637). }

% Manual newpage inserted to improve layout of sample file - not
% needed in general before appendices/bibliography.

%\newpage

%\appendix
%\section*{Appendix A.}
%\label{app:theorem}
%
%% Note: in this sample, the section number is hard-coded in. Following
%% proper LaTeX conventions, it should properly be coded as a reference:
%
%%In this appendix we prove the following theorem from
%%Section~\ref{sec:textree-generalization}:
%
%In this appendix we prove the following theorem from
%Section~6.2:
%
%\noindent
%{\bf Theorem} {\it Let $u,v,w$ be discrete variables such that $v, w$ do
%not co-occur with $u$ (i.e., $u\neq0\;\Rightarrow \;v=w=0$ in a given
%dataset $\dataset$). Let $N_{v0},N_{w0}$ be the number of data points for
%which $v=0, w=0$ respectively, and let $I_{uv},I_{uw}$ be the
%respective empirical mutual information values based on the sample
%$\dataset$. Then
%\[
%	N_{v0} \;>\; N_{w0}\;\;\Rightarrow\;\;I_{uv} \;\leq\;I_{uw}
%\]
%with equality only if $u$ is identically 0.} \hfill\BlackBox
%
%\noindent
%{\bf Proof}. We use the notation:
%\[
%P_v(i) \;=\;\frac{N_v^i}{N},\;\;\;i \neq 0;\;\;\;
%P_{v0}\;\equiv\;P_v(0)\; = \;1 - \sum_{i\neq 0}P_v(i).
%\]
%These values represent the (empirical) probabilities of $v$
%taking value $i\neq 0$ and 0 respectively.  Entropies will be denoted
%by $H$. We aim to show that $\fracpartial{I_{uv}}{P_{v0}} < 0$....\\
%
%{\noindent \em Remainder omitted in this sample. See http://www.jmlr.org/papers/ for full paper.}


\vskip 0.2in
\bibliography{sample}

\end{document}
